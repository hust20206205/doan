

%%%%%%%%%%%%%%%%%%%%%%%%%%%%%%

\chapter*{\centering DANH SÁCH HÌNH ẢNH}

\addcontentsline{toc}{chapter}{DANH SÁCH HÌNH ẢNH}

\makeatletter

\renewcommand\listoffigures{

\@starttoc{lof}

}

\makeatother

\listoffigures

%%%%%%%%%%%%%%%%%%%%%%%%%%%%%%

\chapter*{\centering DANH SÁCH CÁC CỤM TỪ VIẾT TẮT}

\addcontentsline{toc}{chapter}{DANH SÁCH CÁC CỤM TỪ VIẾT TẮT}

% @sau

% @sau

% @sau

% @sau

% @sau

% @sau

% @sau

% @sau

% @sau

% @sau

% @sau

% @sau

% @sau

% @sau

% @sau

% @sau

% @sau

% @sau

% @sau

% @sau

% @sau

% @sau

% @sau

% @sau

% @sau

% @sau

% @sau

% @sau

% @sau

% @sau

\begin{table}[h]

\centering

\begin{tabular}{|c|c|c|c|}

\hline

STT & Từ viết tắt & Từ viết đầy đủ & Mô tả \\

\hline

Dong1 & Dong1 & Cot1 & Cot2 \\

\hline

Dong2 & Dong2 & Cot1 & Cot2 \\

\hline

\end{tabular}

\end{table}

% API; Application Programming Interface; Giao diện lập trình ứng dụng

% CI/CD; Continuous Integration (CI) and Continuous Delivery (CD) ; Quá trình tích hợp và chuyển giao liên tục

% thiết kế hướng miền ; thiết kế hướng miền; Kỹ thuật thiết kế theo hướng miền

% DI; Dependency Injection; Cơ chế tiêm sự phụ thuộc giữa các đối tượng

% HTTP; Hypertext Transfer Protocol; Giao thức truyền tải siêu văn bản

% JSON; JavaScript Object Notation; Một kiểu dữ liệu mở rộng của JavaScript

% ORM; Object Relational Mapping; Một kỹ thuật ánh xạ các đối tượng lập trình với từng bảng trong CSDL quan hệ

% Cơ sở dữ liệu ; CSDL ;

% Tạo (Create), Đọc (Read), Sửa (Update), Xóa (Delete) ; CRUD ;

% Kubernetes ; K8s ; kubernetes

% Số điện thoại ; SĐT ;

% UML

% MVC; Model View Controller; Một mẫu thiết kế ứng dụng

% SQL

SOA; Service Oriented Architecture; Kiến trúc hướng dịch vụ

SOAP; Simple Object Access Protocol; Một giao thức để truy cập dịch vụ web

SPA; Single Page Application; Kiểu ứng dụng một trang

REST; Representational State Transfer; Một tiêu chuẩn thiết kế các API sử dụng cho các dịch vụ web

URL; Uniform Resource Locator ; Địa chỉ định vị tài nguyên trên Internet

XML; Extensible Markup Language; Ngôn ngữ đánh dấu mở rộng

% TCT ; TCT ;

Người nộp thuế ; NNT ;

Mã số thuế ; MST ;

Hóa đơn điện tử ; HĐĐT ;

Cơ quan thuế ; CQT ;

Công nghệ thông tin ; CNTT ;

%%%%%%%%%%%%%%%%%%%%%%%%%%%%%%

\chapter*{\centering DANH SÁCH CÁC THUẬT NGỮ}

\addcontentsline{toc}{chapter}{DANH SÁCH CÁC THUẬT NGỮ}

% @sau

% @sau

% @sau

% @sau

% @sau

% @sau

% @sau

% @sau

% @sau

% @sau

% @sau

% @sau

% @sau

% @sau

% @sau

% @sau

% @sau

% @sau

% @sau

% @sau

% @sau

% @sau

% @sau

% @sau

% @sau

% @sau

% @sau

% @sau

\begin{table}[h]

\centering

\begin{tabular}{|c|c|c|}

\hline

STT & Tiếng Anh & Tiếng Việt \\

\hline

Dong1 & Dong1 & Cot2 \\

\hline

Dong2 & Dong2 & Cot2 \\

\hline

\end{tabular}

\end{table}

% kiến trúc nguyên khối, kiến trúc nguyên khối

% kiến trúc nguyên khối, kiến trúc nguyên khối

% kiến trúc vi dịch, kiến trúc vi dịch

% kiến trúc vi dịch, kiến trúc vi dịch

% kiến trúc vi dịch, kiến trúc vi dịch

% kiến trúc vi dịch, kiến trúc vi dịch

% thiết kế hướng miền, thiết kế hướng miền

% thiết kế hướng miền, thiết kế hướng miền

1 thiết kế hướng miền

Thiết kế hướng lĩnh vực

2 Domain (không dịch)

3 Abstraction Trừu tượng

4 chuyên gia ngành

%%%%%%%%%%%%%%%%%%%%%%%%%%%%%%

\chapter*{\centering MỞ ĐẦU}

\addcontentsline{toc}{chapter}{MỞ ĐẦU}

\section*{Lý do chọn đề tài}

%%%%%%%%%%%%%%%%%%%%%%%%%%%%%%

Trong quá trình hoạt động kinh doanh, doanh nghiệp có nhu cầu chuyển đổi mô hình kinh doanh linh hoạt để có thể tồn tại và phát triển khi thị trường thay đổi. Từ đó, đáp ứng nhu cầu của khách hàng, mang lại ưu thế cạnh tranh so với các đối thủ.

Trong những năm gần đây, việc áp dụng kiến trúc vi dịch vụ ngày càng phổ biến, đem lại nhiều lợi ích như tách các nghiệp vụ kinh doanh thành các dịch vụ nhỏ độc lập, tăng tính linh hoạt và khả năng chống chịu sự cố.

Kiến trúc vi dịch vụ hỗ trợ doanh nghiệp chuyển đổi nhanh chóng để đáp ứng nhu cầu của mô hình kinh doanh và mong đợi của khách hàng. Tuy nhiên, để xây dựng được kiến trúc vi dịch vụ tốt, cần phải tạo ra các dịch vụ nhỏ phù hợp và duy trì tính độc lập. Trong đồ án này, em sử dụng thiết kế hướng miền để phân tích và xây dựng kiến trúc vi dịch vụ.

Theo quy định của Nghị định 123/2020/NĐ - CP, tất cả các doanh nghiệp, tổ chức và hộ kinh doanh đều bắt buộc phải sử dụng hóa điện tử. Vì vậy, nhu cầu sử dụng và xử lý hóa đơn điện tử trở nên rất lớn. Do đó trong đồ án này, em chọn chủ đề \emph{"Sử dụng thiết kế hướng miền xây dựng kiến trúc vi dịch vụ cho bài toán hóa đơn điện tử"}. Chủ đề này là một xu hướng quan trọng trong phát triển phần mềm và mang lại nhiều lợi ích trong việc cải thiện quá trình quản lý hóa đơn điện tử.

%%%%%%%%%%%%%%%%%%%%%%%%%%%%%%

\section*{Đối tượng và phạm vi nghiên cứu}

%%%%%%%%%%%%%%%%%%%%%%%%%%%%%%

\begin{itemize}

\item \textbf{Đối tượng nghiên cứu:} Kiến trúc vi dịch vụ

\item \textbf{Phạm vi nghiên cứu:} Tập trung vào tìm hiểu thiết kế hướng miền xây dựng kiến trúc vi dịch vụ cho bài toán hóa đơn điện tử.

\end{itemize}

%%%%%%%%%%%%%%%%%%%%%%%%%%%%%%

\section*{Tóm tắt nội dung đồ án}

%%%%%%%%%%%%%%%%%%%%%%%%%%%%%%

Báo cáo đồ án này được tổ chức thành các phần chính sau:

\begin{itemize}

\item \textbf{Chương 1: Phân tích thiết kế hệ thống}

\begin{quote}

Nội dung phân tích hệ thống bán hàng.

\end{quote}

\item \textbf{Chương 2: Trình bày về kiến trúc kiến trúc vi dịch vụ}

\begin{quote}

Trình bày các công nghệ, kĩ thuật, nội dung của kiến trúc kiến trúc vi dịch vụ.

\end{quote}

\item \textbf{Chương 3: Các công nghệ đã sử dụng}

\begin{quote}

Trình bày các công nghệ em đã sử dụng trong đồ án này.

\end{quote}

\end{itemize}

%@ Thêm các mục nhỏ như:

%@ Thêm các mục nhỏ như:

%@ Thêm các mục nhỏ như:

%@ Thêm các mục nhỏ như:

%@ Thêm các mục nhỏ như:

%@ Thêm các mục nhỏ như:

%@ Thêm các mục nhỏ như:

%@ Thêm các mục nhỏ như:

%@ Thêm các mục nhỏ như:

%@ Thêm các mục nhỏ như:

%@ Thêm các mục nhỏ như:

%@ Thêm các mục nhỏ như:

%@ Thêm các mục nhỏ như:

% Luận văn được tổ chức thành các phần chính sau:

% Mở đầu: Trình bày tổng quan về đề tài

% Chương 1: Trình bày cách thức phát triển phần mềm theo kiến trúc kiến trúc vi dịch vụ .

% Trong chương này, luận văn tập trung làm rõ các nội dung:

% - Sơ lược về một số hướng kiến trúc phần mềm truyền thống như kiến trúc nguyên

% khối, kiến trúc hướng dịch vụ, công nghệ ESB

% - Tổng quan về kiến trúc kiến trúc vi dịch vụ : sự ra đời, đặc điểm của kiến trúc vi dịch vụ

% - Các mẫu thiết kế quan trọng được sử dụng trong kiến trúc vi dịch vụ

% - Một số nguyên tắc thiết kế kiến trúc vi dịch vụ

% Chương 2: Trình bày hướng xây dựng ứng dụng web sử dụng micro - frontends.

% Trong chương này, luận văn tập trung làm rõ các nội dung:

% - Sơ lược về một số mô hình phát triển web như mô hình web tĩnh, mô hình web

% động, mô hình web theo hướng SPA

% - Sự ra đời của kiến trúc micro - frontends

% - Các cơ chế tích hợp micro - frontends được thảo luận như: tích hợp theo hướng

% “build - time”, tích hợp theo hướng “run - time”, cách thức điều hướng và giao tiếp

% giữa các micro - frontends

% Chương 3: Trình bày cách thức xây dựng một ứng dụng thử nghiệm sử dụng kiến

% trúc kiến trúc vi dịch vụ, micro - frontends. Một số nội dung chính trong quá trình thực nghiệm

% được làm rõ bao gồm:

% 3

% - Áp dụng phương pháp thiết kế hướng miền để phân hoạch, thiết kế chương trình

% - Thiết kế và cài đặt tầng dịch vụ theo hướng kiến trúc vi dịch vụ, sử dụng các công

% nghệ trên nền tảng Java như Spring Boot, Spring Cloud

% - Thiết kế và cài đặt tầng giao diện theo hướng micro - frontends, sử dụng các công

% nghệ như Single - SPA, Angular, ReactJS

% - Một số kỹ thuật kiểm thử kiến trúc vi dịch vụ cũng được thảo luận như kiểm thử đơn

% vị, kiểm thử tích hợp và kiểm thử mức giao diện

% - Cách thức triển khai ứng dụng sử dụng Docker

% Phần kết luận: Tổng kết, đánh giá kết quả thu được của quá trình nghiên cứu cũng

% như các ưu nhược điểm, các hạn chế và hướng phát triển tương lai.

%%%%%%%%%%%%%%%%%%%%%%%%%%%%%%

