\chapter{DDD}

\section{Giới thiệu về thiết kế hướng miền (Domain Driven Design)}

Thiết kế hướng miền được Eric Evans giới thiệu trong cuốn sách \emph{"Domain Driven Design: Tackling Complexity in the Heart of Software"}. \emph{Thiết kế hướng miền (Domain Driven Design)} là một hướng tiếp cận thiết kế phần mềm tập trung vào việc hiểu rõ và mô hình hóa lĩnh vực kinh doanh của một tổ chức. Thiết kế hướng miền nhấn mạnh việc sử dụng lĩnh vực nghiệp vụ kinh doanh để thảo luận và đề xuất giải pháp đáp ứng nhu cầu.

Hệ thống phần mềm được tạo ra để xử lý công việc trong cuộc sống hiện đại. Việc phát triển hệ thống liên kết chặt chẽ với một số khía cạnh cụ thể trong cuộc sống của chúng ta. Trong thiết kế hướng miền, \emph{miền (Domain)} đề cập đến phạm vi kiến thức và vấn đề cụ thể mà hệ thống xử lý.

\begin{itemize}

\item Về góc độ kinh doanh: Miền đại diện cho một lĩnh vực hoặc ngành mà doanh nghiệp hoạt động.

\item Về góc độ hệ thống: Miền có thể coi là đại diện cho không gian vấn đề của hệ thống.

\end{itemize}

\begin{example} \emph{Trong đồ án này, miền được xác định là bài toán giải pháp hóa đơn điện tử.}

\end{example}

Với nhiều phần mềm thiết kế không tốt, phần xử lý các công việc không liên quan đến vấn đề nghiệp vụ kinh doanh như truy cập tập tin, hạ tầng mạng, cơ sở dữ liệu, \dots được lập trình trong đối tượng nghiệp vụ kinh doanh. Cách này giúp tốc độ hoàn thiện nhanh. Tuy nhiên, cách này làm dự án bị mất đi tính hướng đối tượng khó thay thay đổi, mở rộng hệ thống,\dots Thiết kế hướng miền cung cấp một cách để tổ chức mã nguồn và dễ dàng thích ứng với các yêu cầu thay đổi.

Trong kiến trúc vi dịch vụ, thiết kế hướng miền đảm bảo mỗi dịch vụ được thiết kế phản ánh một phần cụ thể của lĩnh vực kinh doanh. Mỗi dịch vụ được quản lí bởi một nhóm phát triển được hỗ trợ bởi các chuyên gia ngành. Trong thiết kế hướng miền, \emph{chuyên gia ngành (Domain Expert)} là người có kiến thức và hiểu biết sâu sắc về vấn đề đang được hệ thống phần mềm giải quyết. Chuyên gia ngành thể hiện chính xác vấn đề kinh doanh, đóng vai trò là nguồn thông tin cho nhóm phát triển.

\section{Cốt lõi của thiết kế hướng miền}

% Giới thiệu về các mẫu chiến lược và các mẫu kỹ thuật

Thiết kế hướng miền cung cấp 2 loại mẫu:

\begin{itemize}

\item \emph{Các mẫu chiến lược (Strategic Patterns):} Phân chia một miền lớn và phức tạp thành các phần nhỏ hơn với ranh giới được xác định rõ ràng. Giúp phân chia một miền lớn hợp lý.

\item \emph{Các mẫu kỹ thuật (Tactical Patterns):} Hiện thực hóa các khái niệm và qui trình trong thành phần thành các thiết kế hệ thống phần mềm. Giúp hệ thống phù hợp với kinh doanh.

\end{itemize}

\begin{figure}[H]

\centering

\includegraphics[scale = 0.5]{pictures/cac_mau_chien_luoc_va_cac_mau_ky_thuat/main.drawio.png}

\caption{Tổng quan về Strategic Patterns và Tactical Patterns}

\end{figure}

Thiết kế hướng miền cung cấp 2 loại mẫu:

\begin{itemize}

\item \emph{Các mẫu chiến lược (Strategic Patterns):} Phân chia một miền lớn và phức tạp thành các phần nhỏ hơn với ranh giới được xác định rõ ràng. Giúp phân chia một miền lớn hợp lý.

\item \emph{Các mẫu kỹ thuật (Tactical Patterns):} Hiện thực hóa các khái niệm và qui trình trong thành phần thành các thiết kế hệ thống phần mềm. Giúp hệ thống phù hợp với kinh doanh.

\end{itemize}

\begin{figure}[H]

\centering

\includegraphics[scale = 0.5]{pictures/cac_mau_chien_luoc_va_cac_mau_ky_thuat/main.drawio.png}

\caption{Tổng quan về Strategic Patterns và Tactical Patterns}

\end{figure}

% \subsection{Chi tiết về các mẫu chiến lược và các mẫu kỹ thuật}

\section{Các mẫu chiến lược của thiết kế hướng miền}

% \chapter{Các mẫu chiến lược}

Các mẫu chiến lược phân tích nghiệp vụ kinh doanh sau đó đưa ra việc phân chia các thành phần và hiểu mối quan hệ của các thành phần đó. Từ đó, các mẫu chiến lược giúp xác định các thành phần quan trọng của hệ thống, đảm bảo kiến trúc phần mềm phản ánh đúng các yêu cầu kinh doanh. Từ việc phân chia hệ thống thành các thành phần nhỏ, chúng ta có thể tạo ra hệ thống mở rộng dễ dàng, phát triển linh hoạt theo nhu cầu kinh doanh.

Các mẫu chiến lược bao gồm:

\begin{itemize}

% các mục nhỏ ben dưới

% các mục nhỏ ben dưới

% các mục nhỏ ben dưới

% các mục nhỏ ben dưới

% các mục nhỏ ben dưới

% các mục nhỏ ben dưới

\item Muc1

\item Muc2

\item Muc1

\item Muc2

\item Muc1

\item Muc2

\item Muc1

\item Muc2

\end{itemize}

% nội dung trang lớn lên để hết giấy

% nội dung trang lớn lên để hết giấy

% nội dung trang lớn lên để hết giấy

% nội dung trang lớn lên để hết giấy

% nội dung trang lớn lên để hết giấy

\begin{figure}[H]

\centering

\includegraphics[scale = 0.9]{pictures/cac_mau_chien_luoc/temp.png}

\caption{Sơ đồ về các thành phần trong mô hình chiến lược}

\end{figure}

%!<! - - $ Vẽ lại sau: - - >

%!<! - - $ Vẽ lại sau: - - >

%!<! - - $ Vẽ lại sau: - - >

%!<! - - $ Vẽ lại sau: - - >

%!<! - - $ Vẽ lại sau: - - >

%!<! - - $ Vẽ lại sau: - - >

%!<! - - $ Vẽ lại sau: - - >

%!<! - - $ Vẽ lại sau: - - >

%!<! - - $ Vẽ lại sau: - - >

%!<! - - $ Vẽ lại sau: - - >

% \section{Miền phụ (Sub - Domain)}

Một miền lớn được tạo thành từ nhiều \emph{miền phụ (Sub - Domain)}. Trong thực tế, một miền kinh doanh phức tạp không thể có một chuyên gia ngành có kiến thức về tất cả các miền phụ.

\begin{example} Trong miền thương mại điện tử lớn có thể có một số miền phụ sau:

\begin{itemize}

\item \textbf{Miền phụ quản lý hàng tồn kho:} liên quan đến việc quản lý sản phẩm trong kho hàng.

\item \textbf{Miền phụ quản lý khách hàng:} liên quan đến việc quản lý tài khoản khách hàng.

\item \textbf{Miền phụ vận chuyển:} liên quan đến việc quản lý việc vận chuyển giao hàng.

\end{itemize}

\end{example}

% \subsection{Phân loại các miền phụ}

Trong thiết kế hướng miền, có ba loại miền phụ là:

\begin{itemize}

\item Miền phụ chung (Generic Subdomain)

\item Miền phụ cốt lõi (Core Subdomain)

\item Miền phụ hỗ trợ (Supporting Subdomain)

\end{itemize}

% \subsubsection{Miền phụ chung (Generic Subdomain)}

Miền phụ chung cung cấp các giải pháp có sẵn mà doanh nghiệp có thể mua. Miền phụ chung có thể được tìm thấy trên nhiều ngành. Doanh nghiệp không thể đạt được bất kỳ lợi thế cạnh tranh nào so với đối thủ bằng cách thực hiện những điều khác biệt trong miền phụ chung.

\begin{example} Các miền phụ chung \textit{"quản lý nhân sự"} hay \textit{"quản lý cơ sở vật chất"} không tạo thêm bất kỳ giá trị khác biệt nào cho doanh nghiệp.

\end{example}

% \subsubsection{Miền phụ cốt lõi (Core Subdomain)}

Miền phụ cốt lõi là phần quan trọng và có giá trị nhất của hệ thống. Miền phụ cốt lõi giúp phân biệt các doanh nghiệp và làm cho các doanh nghiệp có giá trị. Miền phụ cốt lõi tập trung vào mục tiêu và yêu cầu của khách hàng với doanh nghiệp, từ đó quyết định sự thành công của doanh nghiệp. Vì vậy, mỗi doanh nghiệp luôn tìm cách thực hiện những điều khác biệt trong các miền phụ cốt lõi này để đạt được lợi thế so với đối thủ cạnh tranh.

\begin{example} Trong miền thẻ tín dụng, miền phụ cốt lõi có thể là \textit{"phát hành thẻ"} chịu trách nhiệm về quá trình phát hành thẻ tín dụng cho khách hàng. Miền phụ cốt lõi này bao gồm các nhiệm vụ như: thu thập thông tin khách hàng, thực hiện kiểm tra tín dụng, kích hoạt thẻ, \dots

\end{example}

% \subsubsection{Miền phụ hỗ trợ (Supporting Subdomain)}

Các miền phụ cốt lõi phụ thuộc vào các miền phụ hỗ trợ. Miền phụ hỗ trợ cung cấp các dịch vụ để miền phụ cốt lõi hoạt động hiệu quả. Tuy nhiên, miền phụ hỗ trợ không đòi hỏi mức độ phức tạp cao về logic nghiệp vụ.

\begin{example} Trong nhiều phần mềm, miền phụ hỗ trợ \textit{"xác thực người dùng"} OAuth 2.0 của Facebook hoặc Google hỗ trợ cho miền phụ cốt lõi hoạt động hiệu quả.

\end{example}

% \subsection{Cách xác định các miền phụ}

Các miền phụ cốt lõi, hỗ trợ và chung có thể khác nhau đối với các doanh nghiệp hoạt động trong cùng một miền. Vì các miền phụ được xác định tùy theo nhu cầu kinh doanh và bối cảnh cụ thể của mỗi tổ chức.

\subsubsection{Mô tả cách xác định các miền phụ}

\begin{enumerate}

\item Bắt đầu bằng cách xem xét nghiệp vụ kinh doanh.

\item Nếu có sẵn giải pháp đã biết thì có khả năng là miền phụ chung. Ngược lại, chúng ta kiểm tra nghiệp vụ kinh doanh đó có thêm giá trị kinh doanh nào hay không?

\item Nếu không có giá trị kinh doanh thì chúng ta kiểm tra xem các miền phụ cốt lõi có phụ thuộc vào miền phụ này hay không? Nếu có thì có khả năng là miền phụ hỗ trợ. Nếu không thì đó là miền phụ chung.

\item Nếu miền phụ có tiềm năng bổ sung một số giá trị kinh doanh thì bước kiểm tra tiếp theo là xem liệu miền doanh nghiệp có độ phức tạp cao hay không?

\item Nếu miền doanh nghiệp không có độ phức tạp cao thì có khả năng là miền phụ hỗ trợ. Ngược lại thì nó có khả năng là miền phụ cốt lõi.

\end{enumerate}

\begin{figure}[h]

\centering

\includegraphics[scale = 0.5]{pictures/cach_xac_dinh_cac_mien_phu/main.drawio.png}

\caption{Sơ đồ xác định các miền phụ}

\end{figure}

% \subsection{Áp dụng phân loại miền phụ trong đồ án này}

% %!<! - - Hướng dẫn: 5/3 - - >

% %!<! - - Hướng dẫn: 5/3 - - >

% %!<! - - Hướng dẫn: 5/3 - - >

% %!<! - - Hướng dẫn: 5/3 - - >

% %!<! - - Hướng dẫn: 5/3 - - >

% %!<! - - Hướng dẫn: 5/3 - - >

% %!<! - - Hướng dẫn: 5/3 - - >

% %!<! - - Hướng dẫn: 5/3 - - >

% %!<! - - Hướng dẫn: 5/3 - - >

% %!<! - - Hướng dẫn: 5/3 - - >

% %!<! - - Hướng dẫn: 5/3 - - >

% %!<! - - Hướng dẫn: 5/3 - - >

% %!<! - - Hướng dẫn: 5/3 - - >

% %!<! - - Hướng dẫn: 5/3 - - >

% %!<! - - Hướng dẫn: 5/3 - - >

% %!<! - - Hướng dẫn: 5/3 - - >

% ChatGPT?

% Ứng dụng thiết kế hướng miền với hóa đơn điện tử thì miền phụ hỗ trợ có thể là gì?

\subsubsection{Áp dụng phân loại miền phụ trong đồ án này}

\subsubsection{Áp dụng phân loại miền phụ trong đồ án này}

\subsubsection{Áp dụng phân loại miền phụ trong đồ án này}

\subsubsection{Áp dụng phân loại miền phụ trong đồ án này}

\subsubsection{Áp dụng phân loại miền phụ trong đồ án này}

\subsubsection{Áp dụng phân loại miền phụ trong đồ án này}

\subsubsection{Áp dụng phân loại miền phụ trong đồ án này}

\subsubsection{Áp dụng phân loại miền phụ trong đồ án này}

\subsubsection{Áp dụng phân loại miền phụ trong đồ án này}

\subsubsection{Áp dụng phân loại miền phụ trong đồ án này}

\subsubsection{Áp dụng phân loại miền phụ trong đồ án này}

\subsubsection{Áp dụng phân loại miền phụ trong đồ án này}

\subsubsection{Áp dụng phân loại miền phụ trong đồ án này}

\section{Các mẫu kỹ thuật của thiết kế hướng miền}

\section{DDD}

\section{DDD}

\section{DDD}

\section{DDD}

\section{DDD}

% \section{Mô hình miền (Domain Models)}

Để tạo một phần mềm tốt, chúng ta cần phải hiểu rõ về phần mềm đó. Trong thiết kế hướng miền để có thể hiểu miền nhanh, chúng ta cần tạo ra các mô hình miền. Mô hình miền (Domain Models) là kiến thức có tổ chức và có cấu trúc về miền phù hợp để giải quyết vấn đề kinh doanh. Mục tiêu của mô hình miền là cung cấp rõ ràng, ngắn gọn và chính xác về miền làm cơ sở để hệ thống giải quyết vấn đề kinh doanh.

% \begin{example} Trong đồ án này, mô hình miền của em bao gồm các yêu cầu nghiệp vụ và các sơ đồ: UML Use Case Diagrams, UML Class Diagrams,\dots kĩ thuật ở phần sau

% \end{example}

% \section{Bối cảnh bị giới hạn (Bounded Context)}

Một miền cần chia đủ nhỏ để phù hợp với một nhóm cụ thể. Để đạt được điều này, chúng ta cần xác định rõ ranh giới giữa các ngữ cảnh. \emph{Bối cảnh bị giới hạn (Bounded Context)} giúp xác định rõ các ranh giới, chia miền thành các phần độc lập để giải quyết sự phức tạp trong mô hình doanh nghiệp. Bối cảnh bị giới hạn tạo ra các mô hình khác nhau cho các lĩnh vực khác nhau của miền. Bối cảnh bị giới hạn thể hiện phạm vi kinh doanh của dịch vụ.

\begin{figure}[H]

\centering

\includegraphics[scale = 1]{pictures/boi_canh_gioi_han/main.png}

\caption{Ví dụ về bối cảnh bị giới hạn trong một ngân hàng}

\end{figure}

\subsubsection{Cách xác định bối cảnh bị giới hạn}

Để có thể xác định được bối cảnh bị giới hạn chúng ta có thể xem xét:

\begin{itemize}

\item Dựa vào việc phân chia các miền phụ.

\item Dựa vào sơ đồ cấu trúc tổ chức các phòng ban của doanh nghiệp.

\item Dựa vào modules của các ứng dụng kiến trúc nguyên khối (nếu việc phân chia tốt).

\item Dựa vào trách nhiệm và hoạt động của chuyên gia ngành.

\end{itemize}

\subsubsection{Áp dụng xác định bối cảnh bị giới hạn trong đồ án này}

\subsubsection{Áp dụng xác định bối cảnh bị giới hạn trong đồ án này}

\subsubsection{Áp dụng xác định bối cảnh bị giới hạn trong đồ án này}

\subsubsection{Áp dụng xác định bối cảnh bị giới hạn trong đồ án này}

\subsubsection{Áp dụng xác định bối cảnh bị giới hạn trong đồ án này}

\subsubsection{Áp dụng xác định bối cảnh bị giới hạn trong đồ án này}

\subsubsection{Áp dụng xác định bối cảnh bị giới hạn trong đồ án này}

\subsubsection{Áp dụng xác định bối cảnh bị giới hạn trong đồ án này}

\subsubsection{Áp dụng xác định bối cảnh bị giới hạn trong đồ án này}

\subsubsection{Áp dụng xác định bối cảnh bị giới hạn trong đồ án này}

\subsubsection{Áp dụng xác định bối cảnh bị giới hạn trong đồ án này}

\subsubsection{Áp dụng xác định bối cảnh bị giới hạn trong đồ án này}

%!<! - - Hướng dẫn 5/10 - - >

%!<! - - Hướng dẫn 5/10 - - >

%!<! - - Hướng dẫn 5/10 - - >

%!<! - - Hướng dẫn 5/10 - - >

%!<! - - Hướng dẫn 5/10 - - >

%!<! - - Hướng dẫn 5/10 - - >

%!<! - - Hướng dẫn 5/10 - - >

%!<! - - Hướng dẫn 5/10 - - >

%!<! - - Hướng dẫn 5/10 - - >

% \section{Ngôn ngữ chung (Ubiquitous Language)}

Trong quá trình xây dựng mô hình miền, cần có trao đổi giữa người thiết kế hệ thống và chuyên gia ngành để hiểu đúng về miền. Tuy nhiên, nhóm kinh doanh sử dụng ngôn ngữ kinh doanh và nhóm công nghệ có xu hướng sử dụng các thuật ngữ kỹ thuật trong giao tiếp của họ. Người phát triển phần mềm tập trung vào lớp, phương thức, thuật toán, trong khi chuyên gia ngành thường sử dụng ngôn ngữ chuyên ngành của họ. Sự khác biệt về ngôn ngữ giữa các thành viên có thể dẫn đến những thách thức về giao tiếp. Ngoài ra, trong các lĩnh vực kinh doanh khác nhau, một thuật ngữ có thể được sử dụng trong nhiều miền, cùng với ý nghĩa khác nhau gây ra sự nhầm lẫn và hiểu sai cho các người phát triển phần mềm cũng như các chuyên gia ngành.

Thiết kế hướng miền đề xuất sử dụng ngôn ngữ chung để giải quyết những thách thức ngôn ngữ. \emph{Ngôn ngữ chung (Ubiquitous Language)} là một ngôn ngữ được cấu trúc xung quanh mô hình miền và được tất cả các thành viên trong nhóm sử dụng cho mọi hoạt động của nhóm với phần mềm. Ngôn ngữ chung được xác định bởi các từ vựng và có định nghĩa rõ ràng về ngữ cảnh sử dụng từ vựng.

\subsubsection{Một số đặc điểm của ngôn ngữ chung}

\begin{itemize}

\item Ngôn ngữ chung được sử dụng bởi cả chuyên gia ngành và chuyên gia công nghệ.

\item Có nhiều ngôn ngữ chung trong một tổ chức được mỗi nhóm tạo và quản lý một cách độc lập.

\item Việc tạo ra ngôn ngữ chung là một quá trình liên tục. Ngôn ngữ chung phát triển theo thời gian thông qua sự cộng tác giữa doanh nghiệp và các chuyên gia công nghệ.

\item Các thành viên phải sử dụng ngôn ngữ chung cho công việc và trong toàn bộ hệ thống

\end{itemize}

\begin{figure}[H]

\centering

\includegraphics[scale = 0.6]{pictures/ngon_ngu_chung/main.png}

\caption{Ngôn ngữ chung được sử dụng trong toàn bộ hệ thống}

\end{figure}

%@ Tích hợp liên tục

%@ Tích hợp liên tục

%@ Tích hợp liên tục

%@ Tích hợp liên tục

%@ Tích hợp liên tục

%!<! - - @Tích hợp Liên tục (Continuous Integration) - - >

Tích hợp Liên tục (Continuous Integration): là việc các thành viên trong nhóm phát triển tích hợp mã nguồn vào một hệ thống chung thường xuyên. Khi có mã nguồn mới việc tích hợp liên tục sẽ tự động kiểm thử và xây dựng giảm xung đột giữa các phiên bản mã nguồn khác nhau, giúp phát hiện và sửa lỗi sớm hơn.

% CD là gì

% CD là gì

% CD là gì

% CD là gì

% CD là gì

% CD là gì

% CD là gì

% CD là gì

% CD là gì

CI/CD là viết tắt của hai khái niệm quan trọng trong quá trình phát triển phần mềm: Continuous Integration (CI) và Continuous Delivery (CD). Đây là một phương pháp giúp tự động hóa quy trình phát triển, kiểm thử, và triển khai ứng dụng, giúp tăng cường chất lượng phần mềm và giảm thời gian cũng như rủi ro trong quá trình phát triển.

1. **Continuous Integration (CI - Tích hợp liên tục):**

- **Mục tiêu:** Đảm bảo rằng mã nguồn mới được tích hợp vào mã nguồn chính (main codebase) một cách tự động và thường xuyên, giảm thời gian giữa việc viết mã và phát hành.

- **Quy trình:** Mỗi khi một nhà phát triển hoàn thành một tính năng hoặc sửa lỗi, họ tích hợp mã của mình vào mã nguồn chính. Hệ thống CI sẽ tự động kiểm tra mã này bằng cách chạy các bài kiểm thử tự động để đảm bảo rằng nó không làm hỏng hệ thống.

2. **Continuous Delivery (CD - Phân phối liên tục):**

- **Mục tiêu:** Tự động hóa việc triển khai ứng dụng để có thể phân phối bản vá, tính năng hoặc cập nhật một cách nhanh chóng và đáng tin cậy.

- **Quy trình:** Nếu quá trình CI thành công, mã nguồn sẽ được triển khai tự động lên môi trường thử nghiệm (staging environment). Nếu mọi thứ ổn, nó có thể được triển khai tự động lên môi trường sản phẩm (production environment).

3. **Continuous Deployment (CD - Triển khai liên tục):**

- **Khác biệt với Continuous Delivery:** Trong Continuous Deployment, nếu mọi thứ qua bài kiểm thử được tự động và thành công, mã nguồn sẽ tự động triển khai lên môi trường sản phẩm mà không cần sự can thiệp thủ công.

- **Mục tiêu:** Tối ưu hóa quá trình triển khai, giảm thiểu sự chờ đợi và đảm bảo tính ổn định của hệ thống.

4. **Các công cụ thường được sử dụng:**

- **Jenkins, Travis CI, CircleCI:** Đối với CI.

- **Docker, Kubernetes:** Đối với CD, đặc biệt là việc triển khai và quản lý containerized applications.

- **Ansible, Puppet, Chef:** Công cụ tự động hóa cấu hình và triển khai.

Tổng cộng, CI/CD giúp tăng cường khả năng linh hoạt, đảm bảo chất lượng mã nguồn, giảm rủi ro, và giảm thời gian giữa việc phát triển và triển khai sản phẩm.

% CD là gì

% CD là gì

% CD là gì

% CD là gì

% CD là gì

% CD là gì

% CD là gì

% CD là gì

% CD là gì

% CD là gì

Khi một bối cảnh giới hạn đã được xác định, chúng ta cần đảm bảo rằng nó luôn ở trạng thái mới và hoạt động tốt như kỳ vọng. Đáp ứng nhu cầu doanh nghiệp phát triển thay đổi liên tục và nhanh chóng.

Khi cùng vận hành và phát triển xung đột có thể xảy ra ở cùng hoặc khác bối cảnh giới hạn.

= > Vì vậy, cần sử dụng việc tích hợp liên tục tạo ra một quy trình tự động và liên tục từ việc tích hợp mã nguồn, kiểm thử tự động giúp tăng cường chất lượng phần mềm, giảm thời gian và rủi ro trong quá trình phát triển phần mềm.

%!<! - - $VD: jenkins - - >

%!<! - - unit test - - >

%!<! - - test tích hợp - - >

% vở

% thời gian khác nhau

% http://localhost nên không có CD

% \section{Bản đồ bối cảnh (Context Maps)}

Các bối cảnh bị giới hạn phải độc lập trong bối cảnh riêng và có mô hình miền riêng, nhưng các bối cảnh bị giới hạn cần tương tác, giao tiếp để trao đổi thông tin. Vì vậy các bối cảnh bị giới hạn có thể có mối quan hệ với nhau. Những mối quan hệ này cần được quản lý chặt chẽ để hoạt động độc lập, nhất quán và linh hoạt. Do đó, cần phải ghi lại các mối quan hệ thông qua việc sử dụng bản đồ bối cảnh. \emph{Bản đồ bối cảnh (Context Maps)} là sự thể hiện trực quan của hệ thống, thể hiện các thành phần và mối quan hệ giữa các thành phần.

\begin{figure}[H]

\centering

\includegraphics[scale = 0.4]{pictures/ban_do_boi_canh/main.drawio.png}

\caption{Ví dụ bản đồ bối cảnh trong 1 ngân hàng}

\end{figure}

%! Vẽ lại tiếng Việt

%! Vẽ lại tiếng Việt

%! Vẽ lại tiếng Việt

%! Vẽ lại tiếng Việt

%! Vẽ lại tiếng Việt

%! Vẽ lại tiếng Việt

%! Vẽ lại tiếng Việt

%! Vẽ lại tiếng Việt

%! Vẽ lại tiếng Việt

%! Vẽ lại tiếng Việt

% \section{Các mối quan hệ bối cảnh bị giới hạn}

Có 3 loại mối quan hệ giữa các bối cảnh bị giới hạn là:

\begin{itemize}

\item Mối quan hệ đối xứng (Symmetric Relationship)

\textbf{Mô tả:} Thể hiện sự tương tác 2 chiều giữa 2 bối cảnh bị giới hạn .

\item Mối quan hệ bất đối xứng (Asymmetric Relationship)

\textbf{Mô tả:} Thể hiện sự tương tác 1 chiều giữa 2 các bối cảnh bị giới hạn .

\item Mối quan hệ 1 - nhiều (One to Many Relationship)

\textbf{Mô tả:} Thể hiện sự tương tác 1 chiều giữa 1 bối cảnh bị giới hạn với nhiều bối cảnh bị giới hạn khác.

\end{itemize}

\begin{figure}[H]

\centering

\includegraphics[scale = 0.5]{pictures/cac_moi_quan_he_boi_canh_gioi_han/main.png}

\caption{Các mối quan hệ bối cảnh bị giới hạn}

\end{figure}

%%%%%%%%%%%%%%%%%%%%%%%%%%%%%%%%%%

%%%%%%%%%%%%%%%%%%%%%%%%%%%%%%%%%%

%%%%%%%%%%%%%%%%%%%%%%%%%%%%%%%%%%

%%%%%%%%%%%%%%%%%%%%%%%%%%%%%%%%%%

%%%%%%%%%%%%%%%%%%%%%%%%%%%%%%%%%%

%%%%%%%%%%%%%%%%%%%%%%%%%%%%%%%%%%

%%%%%%%%%%%%%%%%%%%%%%%%%%%%%%%%%%

% \subsection{Mối quan hệ đối xứng (Symmetric Relationship)}

% \subsubsection{Mô hình riêng biệt (Separate Ways)}

Mô hình riêng biệt (Separate Ways) khi các bối cảnh bị giới hạn có quan hệ riêng biệt, không có sự phụ thuộc. Vì vậy, các bối cảnh bị giới hạn này có ngôn ngữ, mô hình, mục đích độc lập và thực thi riêng biệt. Các nhóm phát triển không phải cộng tác hay phối hợp với nhau từ đó đem lại lợi ích dễ dàng bảo trì và mở rộng hệ thống.

\begin{example} Trong miền vấn đề ngân hàng, thẻ tín dụng và khoản vay mua nhà không có mối quan hệ.

\begin{figure}[H]

\centering

\includegraphics[scale = 0.5]{pictures/mo_hinh_rieng_biet_separate_ways/main.drawio.png}

\caption{Ví dụ mô hình riêng biệt (Separate Ways)}

\end{figure}

\end{example}

% \subsubsection{Mô hình hạt nhân chung (Shared Kernel)}

Trong thực tế, nhiều bối cảnh bị giới hạn phụ thuộc lẫn nhau. Mô hình hợp tác (Partnership) tạo điều kiện cho việc giao tiếp và cộng tác giữa các bối cảnh bị giới hạn phụ thuộc. Tuy nhiên, sự phụ thuộc này dẫn đến mức độ kết hợp cao giữa các nhóm và bối cảnh bị giới hạn, dẫn tới mất đi tính độc lập.

\textit{Lưu ý: Mô hình hợp tác không phải là mô hình của các mẫu chiến lược trong thiết kế huớng miền.}

Để giải quyết vấn đề bối cảnh bị giới hạn phụ thuộc lẫn nhau chúng ta có mô hình hạt nhân chung. Mô hình hạt nhân chung (Shared Kernel) cho phép các bối cảnh bị giới hạn có phần chia sẻ chung và có ranh giới phân định rõ ràng. Từ đó, tách việc quản lí các mô hình hạt nhân chung này một cách độc lập với phần còn lại của bối cảnh bị giới hạn . Khi cần thay đổi mà không phải của mô hình hạt nhân chung thì nhóm sẽ hoạt động độc lập. Thông thường, mô hình hạt nhân chung được hiện thực hóa bằng các thư viện chung. Tuy nhiên, chỉ sử dụng mô hình hạt nhân chung nếu quan hệ của các bối cảnh bị giới hạn nhỏ và ổn định để tránh quan hệ phức tạp và ràng buộc chặt chẽ.

% Vẽ lại bản đồ tiếng Việt

% Vẽ lại bản đồ tiếng Việt

% Vẽ lại bản đồ tiếng Việt

% Vẽ lại bản đồ tiếng Việt

% Vẽ lại bản đồ tiếng Việt

% Vẽ lại bản đồ tiếng Việt

% Vẽ lại bản đồ tiếng Việt

% Vẽ lại bản đồ tiếng Việt

% Từ bản đồ lấy vi dụ cho các mô hình

% Từ bản đồ lấy vi dụ cho các mô hình

% Từ bản đồ lấy vi dụ cho các mô hình

% Từ bản đồ lấy vi dụ cho các mô hình

% Từ bản đồ lấy vi dụ cho các mô hình

% Từ bản đồ lấy vi dụ cho các mô hình

% Từ bản đồ lấy vi dụ cho các mô hình

% Từ bản đồ lấy vi dụ cho các mô hình

% Từ bản đồ lấy vi dụ cho các mô hình

% Từ bản đồ lấy vi dụ cho các mô hình

\begin{example} Trong miền vấn đề ngân hàng, thẻ tín dụng và khoản vay mua nhà không có mối quan hệ.

\begin{figure}[H]

\centering

\includegraphics[scale = 0.5]{pictures/mo_hinh_rieng_biet_separate_ways/main.drawio.png}

\caption{Ví dụ mô hình riêng biệt (Separate Ways)}

\end{figure}

\end{example}

% %! $VD: hình giao như 2 tập hợp - - >

% \subsection{Mối quan hệ bất đối xứng (Asymmetric Relationship)}

Trong mối quan hệ bất đối xứng, một bối cảnh bị giới hạn có sự phụ thuộc vào một bối cảnh bị giới hạn khác. Mối quan hệ này được mô tả bằng cách gán vai trò cho bối cảnh bị giới hạn :

\begin{itemize}

\item \textbf{Bối cảnh bị giới hạn thượng nguồn (Upstream):}

\begin{itemize}

\item Bối cảnh bị giới hạn cung cấp cho bối cảnh bị giới hạn khác.

\item Ký hiệu: U

\end{itemize}

\item \textbf{Bối cảnh bị giới hạn hạ lưu (Downstream):}

\begin{itemize}

\item Bối cảnh bị giới hạn phụ thuộc vào bối cảnh bị giới hạn khác.

\item Ký hiệu: D

\end{itemize}

\end{itemize}

\begin{example} Mối quan hệ bất đối xứng giữa bối cảnh bị giới hạn A và bối cảnh bị giới hạn B.

\begin{itemize}

\item Bối cảnh bị giới hạn A ràng buộc với bối cảnh bị giới hạn B

\item Bối cảnh bị giới hạn A đóng vai trò là bối cảnh bị giới hạn hạ lưu (Downstream)

\item Bối cảnh bị giới hạn B đóng vai trò là bối cảnh bị giới hạn thượng nguồn (Upstream)

\item Bối cảnh bị giới hạn A có kiến thức về các mô hình trong bối cảnh bị giới hạn B

\item Bối cảnh bị giới hạn B không có bất kỳ kiến thức nào về mô hình trong bối cảnh bị giới hạn A

\end{itemize}

\begin{figure}[H]

\centering

\includegraphics[scale = 0.5]{pictures/moi_quan_he_bat_doi_xung/main.drawio.png}

\caption{Ví dụ mối quan hệ bất đối xứng}

\end{figure}

\end{example}

% \subsubsection{Mô hình khách hàng - nhà cung cấp (Customer - Supplier)}

Mô hình khách hàng - nhà cung cấp (Customer - Supplier) được thể hiện rằng bối cảnh bị giới hạn thượng nguồn đáp ứng nhu cầu của bối cảnh bị giới hạn hạ lưu.

Khi đó:

\begin{itemize}

\item Bối cảnh bị giới hạn thượng nguồn được gọi là nhà cung cấp.

\item Bối cảnh bị giới hạn hạ lưu được gọi là khách hàng.

\end{itemize}

Trong thực tế, nhóm phát triển nhà cung cấp luôn tham khảo ý kiến của nhóm phát triển khách hàng và có bộ kiểm thử để đảm bảo rằng dịch vụ của nhà cung cấp đáp ứng được yêu cầu của khách hàng.

% \subsubsection{Mô hình tuân thủ (Conformist)}

Trong mô hình khách hàng - nhà cung cấp, nếu nhà cung cấp thực hiện tốt yêu cầu thì khách hàng cần tuân thủ chặt chẽ. Mô hình tuân thủ (Conformist) là một mối quan hệ trong đó bối cảnh bị giới hạn hạ lưu áp dụng mô hình, ngôn ngữ chung và các khái niệm của bối cảnh bị giới hạn thượng nguồn.

Trong mô hình tuân thủ bối cảnh bị giới hạn hạ lưu được ký hiệu là CF.

%! $VD: - - >

%! $VD: A - CF - U - B - - >

%! $VD: A - users(id, name) - B cũng users(id, name) - - >

% Vẽ lại bản đồ tiếng Việt

% Vẽ lại bản đồ tiếng Việt

% Vẽ lại bản đồ tiếng Việt

% Vẽ lại bản đồ tiếng Việt

% Vẽ lại bản đồ tiếng Việt

% Vẽ lại bản đồ tiếng Việt

% Vẽ lại bản đồ tiếng Việt

% Vẽ lại bản đồ tiếng Việt

% Từ bản đồ lấy vi dụ cho các mô hình

% Từ bản đồ lấy vi dụ cho các mô hình

% Từ bản đồ lấy vi dụ cho các mô hình

% Từ bản đồ lấy vi dụ cho các mô hình

% Từ bản đồ lấy vi dụ cho các mô hình

% Từ bản đồ lấy vi dụ cho các mô hình

% Từ bản đồ lấy vi dụ cho các mô hình

% Từ bản đồ lấy vi dụ cho các mô hình

% Từ bản đồ lấy vi dụ cho các mô hình

% Từ bản đồ lấy vi dụ cho các mô hình

\begin{example} Trong miền vấn đề ngân hàng, thẻ tín dụng và khoản vay mua nhà không có mối quan hệ.

\begin{figure}[H]

\centering

\includegraphics[scale = 0.5]{pictures/mo_hinh_rieng_biet_separate_ways/main.drawio.png}

\caption{Ví dụ mô hình riêng biệt (Separate Ways)}

\end{figure}

\end{example}



% \subsubsection{Mô hình chống đổ vỡ (Anti Corruption Layer)}

Trong mô hình khách hàng - nhà cung cấp, nếu nhà cung cấp có thể thay đổi linh hoạt không đảm bảo đáp ứng nhu cầu của khách hàng thì khách hàng cần có giải pháp xử lí. Mô hình chống đổ vỡ (Anti Corruption Layer) là một mối quan hệ trong đó bối cảnh bị giới hạn hạ lưu sử dụng một lớp để dịch giữa ngôn ngữ của nó và ngôn ngữ của bối cảnh bị giới hạn thượng nguồn.

Trong mô hình chống đổ vỡ, mỗi bối cảnh bị giới hạn có mô hình riêng biệt và lớp chống đổ vỡ cần kiến thức về mô hình hạ lưu và thượng nguồn để bảo vệ hạ lưu và duy trì tính toàn vẹn.

%@ Façade

%@ Adapter

Trong mô hình chống đổ vỡ bối cảnh bị giới hạn hạ lưu được ký hiệu là ACL.

% Vẽ lại bản đồ tiếng Việt

% Vẽ lại bản đồ tiếng Việt

% Vẽ lại bản đồ tiếng Việt

% Vẽ lại bản đồ tiếng Việt

% Vẽ lại bản đồ tiếng Việt

% Vẽ lại bản đồ tiếng Việt

% Vẽ lại bản đồ tiếng Việt

% Vẽ lại bản đồ tiếng Việt

% Từ bản đồ lấy vi dụ cho các mô hình

% Từ bản đồ lấy vi dụ cho các mô hình

% Từ bản đồ lấy vi dụ cho các mô hình

% Từ bản đồ lấy vi dụ cho các mô hình

% Từ bản đồ lấy vi dụ cho các mô hình

% Từ bản đồ lấy vi dụ cho các mô hình

% Từ bản đồ lấy vi dụ cho các mô hình

% Từ bản đồ lấy vi dụ cho các mô hình

% Từ bản đồ lấy vi dụ cho các mô hình

% Từ bản đồ lấy vi dụ cho các mô hình

\begin{example} Trong miền vấn đề ngân hàng, thẻ tín dụng và khoản vay mua nhà không có mối quan hệ.

\begin{figure}[H]

\centering

\includegraphics[scale = 0.5]{pictures/mo_hinh_rieng_biet_separate_ways/main.drawio.png}

\caption{Ví dụ mô hình riêng biệt (Separate Ways)}

\end{figure}

\end{example}



% \subsection{Mối quan hệ 1 - nhiều (One to Many Relationship)}

% \subsubsection{Dịch vụ máy chủ mở (Open Host Service)}
Dịch vụ máy chủ mở (Open Host Service) là nhà cung cấp trong mô hình khách hàng - nhà cung cấp, dịch vụ máy chủ mở hiển thị một API công khai cho các bối cảnh bị giới hạn khác sử dụng chức năng của nhà cung cấp.

Trong bản đồ bối cảnh, dịch vụ máy chủ mở được ký hiệu là OHS.

% Vẽ lại bản đồ tiếng Việt

% Vẽ lại bản đồ tiếng Việt

% Vẽ lại bản đồ tiếng Việt

% Vẽ lại bản đồ tiếng Việt

% Vẽ lại bản đồ tiếng Việt

% Vẽ lại bản đồ tiếng Việt

% Vẽ lại bản đồ tiếng Việt

% Vẽ lại bản đồ tiếng Việt

% Từ bản đồ lấy vi dụ cho các mô hình

% Từ bản đồ lấy vi dụ cho các mô hình

% Từ bản đồ lấy vi dụ cho các mô hình

% Từ bản đồ lấy vi dụ cho các mô hình

% Từ bản đồ lấy vi dụ cho các mô hình

% Từ bản đồ lấy vi dụ cho các mô hình

% Từ bản đồ lấy vi dụ cho các mô hình

% Từ bản đồ lấy vi dụ cho các mô hình

% Từ bản đồ lấy vi dụ cho các mô hình

% Từ bản đồ lấy vi dụ cho các mô hình

\begin{example} Trong miền vấn đề ngân hàng, thẻ tín dụng và khoản vay mua nhà không có mối quan hệ.

\begin{figure}[H]

\centering

\includegraphics[scale = 0.5]{pictures/mo_hinh_rieng_biet_separate_ways/main.drawio.png}

\caption{Ví dụ mô hình riêng biệt (Separate Ways)}

\end{figure}

\end{example}



% \subsubsection{Ngôn ngữ được xuất bản (Published Language)}

Khi ngôn ngữ chung ở dịch vụ máy chủ mở được các nhóm phát triển trong bối cảnh bị giới hạn hạ lưu chấp nhận. Ngôn ngữ chung này được gọi là ngôn ngữ được xuất bản (Published Language). Ngôn ngữ được xuất bản có lợi ích là tính thống nhất trong hệ thống tuy nhiên cần phân tích kĩ vì nó có thể tạo ra sự nhầm lẫn trong bối cảnh bị giới hạn hạ lưu nào đó.

Trong bản đồ bối cảnh, ngôn ngữ được xuất bản kết hợp dịch vụ máy chủ mở được ký hiệu là OHS|PL.

% Vẽ lại bản đồ tiếng Việt

% Vẽ lại bản đồ tiếng Việt

% Vẽ lại bản đồ tiếng Việt

% Vẽ lại bản đồ tiếng Việt

% Vẽ lại bản đồ tiếng Việt

% Vẽ lại bản đồ tiếng Việt

% Vẽ lại bản đồ tiếng Việt

% Vẽ lại bản đồ tiếng Việt

% Từ bản đồ lấy vi dụ cho các mô hình

% Từ bản đồ lấy vi dụ cho các mô hình

% Từ bản đồ lấy vi dụ cho các mô hình

% Từ bản đồ lấy vi dụ cho các mô hình

% Từ bản đồ lấy vi dụ cho các mô hình

% Từ bản đồ lấy vi dụ cho các mô hình

% Từ bản đồ lấy vi dụ cho các mô hình

% Từ bản đồ lấy vi dụ cho các mô hình

% Từ bản đồ lấy vi dụ cho các mô hình

% Từ bản đồ lấy vi dụ cho các mô hình

\begin{example} Trong miền vấn đề ngân hàng, thẻ tín dụng và khoản vay mua nhà không có mối quan hệ.

\begin{figure}[H]

\centering

\includegraphics[scale = 0.5]{pictures/mo_hinh_rieng_biet_separate_ways/main.drawio.png}

\caption{Ví dụ mô hình riêng biệt (Separate Ways)}

\end{figure}

\end{example}



%%%%%%%%%%%%%%%%%%%%%%%%%%%%%%%%%%

%%%%%%%%%%%%%%%%%%%%%%%%%%%%%%%%%%

%%%%%%%%%%%%%%%%%%%%%%%%%%%%%%%%%%

%%%%%%%%%%%%%%%%%%%%%%%%%%%%%%%%%%

%%%%%%%%%%%%%%%%%%%%%%%%%%%%%%%%%%

%%%%%%%%%%%%%%%%%%%%%%%%%%%%%%%%%%

%%%%%%%%%%%%%%%%%%%%%%%%%%%%%%%%%%

%%%%%%%%%%%%%%%%%%%%%%%%%%%%%%%%%%

% \section{Áp dụng về các mối quan hệ bối cảnh bị giới hạn}

%! Hướng dẫn 6/6

%%%%%%%%%%%%%%%%%%%%%%%%%%%%%%%%%%

%%%%%%%%%%%%%%%%%%%%%%%%%%%%%%%%%%

%%%%%%%%%%%%%%%%%%%%%%%%%%%%%%%%%%

%%%%%%%%%%%%%%%%%%%%%%%%%%%%%%%%%%

%%%%%%%%%%%%%%%%%%%%%%%%%%%%%%%%%%

%%%%%%%%%%%%%%%%%%%%%%%%%%%%%%%%%%

%%%%%%%%%%%%%%%%%%%%%%%%%%%%%%%%%%

%%%%%%%%%%%%%%%%%%%%%%%%%%%%%%%%%%

\chapter{Các mẫu kỹ thuật}

% \input{contents/cac_mau_ky_thuat}

\section{Đối tượng miền (Domain Object)}

% \input{contents/cac_doi_tuong_mien_domain_object}

\subsection{Đối tượng thực thể (Entities Objects)}

\input{contents/doi_tuong_thuc_the_entities_objects}

\subsection{Đối tượng giá trị (Value Objects)}

% Đối tượng miền hoặc đối tượng giá trị, là đối tượng đại diện cho một đặc điểm của miền mà không có nhận dạng riêng.

% \input{contents/doi_tuong_gia_tri_value_objects}

\subsection{Miền dịch vụ (Service Domain)}

% \input{contents/mien_dich_vu_srv}

%! Hướng dẫn 7/4

%! Hướng dẫn 7/5

% \subsubsection{xxxxxxx}

% \chapter*{\centering MỞ ĐẦU}
\addcontentsline{toc}{chapter}{MỞ ĐẦU}
\section*{Lý do chọn đề tài}
% \input{contents/ly_do_chon_de_tai}
\section*{Đối tượng và phạm vi nghiên cứu}
% \input{contents/doi_tuong_va_pham_vi_nghien_cuu}
\section*{Tóm tắt nội dung đồ án}
% \input{contents/tom_tat_noi_dung_do_an}

%%%%%%%%%%%%%%%%%%%%%%%%%%%%%%%%%%

\end{document} % Kết thúc

% kết luận, tài liệu tham khảo

% \bibliographystyle{plain}

% \bibliography{references}

%%%%%%%%%%%%%%%%%%%%%%%%%%%%%%%%%%

% % %! Aggregates/ /

% % Tổng hợp là đối tượng kinh doanh trung tâm trong Bối cảnh bị giới hạn của chúng ta và xác định phạm vi nhất quán trong bối cảnh bị giới hạn đó.

% % Tổng hợp = Mã định danh chính của Bối cảnh bị giới hạn của chúng ta

% \subsubsection{xxxxxxx}

% % \chapter*{\centering MỞ ĐẦU}
\addcontentsline{toc}{chapter}{MỞ ĐẦU}
\section*{Lý do chọn đề tài}
% \input{contents/ly_do_chon_de_tai}
\section*{Đối tượng và phạm vi nghiên cứu}
% \input{contents/doi_tuong_va_pham_vi_nghien_cuu}
\section*{Tóm tắt nội dung đồ án}
% \input{contents/tom_tat_noi_dung_do_an}

% \end{document} % kết thúc

% Yêu cầu nghiệp vụ của từng sub

% %

% Sơ đồ if else Đ S

% %

% sub trước model

% %

%%%%%%%%%%%%%%%%%%%%%%%%%%%%%%%%%%%%%

\end{document}

\section{xxxxxxx}

\subsection{xxxxxxx}

\subsubsection{xxxxxxx}

\chapter*{\centering MỞ ĐẦU}
\addcontentsline{toc}{chapter}{MỞ ĐẦU}
\section*{Lý do chọn đề tài}
% \input{contents/ly_do_chon_de_tai}
\section*{Đối tượng và phạm vi nghiên cứu}
% \input{contents/doi_tuong_va_pham_vi_nghien_cuu}
\section*{Tóm tắt nội dung đồ án}
% \input{contents/tom_tat_noi_dung_do_an}

