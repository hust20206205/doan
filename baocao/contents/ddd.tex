\chapter{DDD}

\section{Giới thiệu về thiết kế hướng miền (Domain Driven Design)}

Thiết kế hướng miền được Eric Evans giới thiệu trong cuốn sách \emph{"Domain Driven Design: Tackling Complexity in the Heart of Software"}. \emph{Thiết kế hướng miền (Domain Driven Design)} là một hướng tiếp cận thiết kế phần mềm tập trung vào việc hiểu rõ và mô hình hóa lĩnh vực kinh doanh của một tổ chức. Thiết kế hướng miền nhấn mạnh việc sử dụng lĩnh vực nghiệp vụ kinh doanh để thảo luận và đề xuất giải pháp đáp ứng nhu cầu.

Hệ thống phần mềm được tạo ra để xử lý công việc trong cuộc sống hiện đại. Việc phát triển hệ thống liên kết chặt chẽ với một số khía cạnh cụ thể trong cuộc sống của chúng ta. Trong thiết kế hướng miền, \emph{miền (Domain)} đề cập đến phạm vi kiến thức và vấn đề cụ thể mà hệ thống xử lý.

\begin{itemize}

\item Về góc độ kinh doanh: Miền đại diện cho một lĩnh vực hoặc ngành mà doanh nghiệp hoạt động.

\item Về góc độ hệ thống: Miền có thể coi là đại diện cho không gian vấn đề của hệ thống.

\end{itemize}

\begin{example} \emph{Trong đồ án này, miền được xác định là bài toán giải pháp hóa đơn điện tử.}

\end{example}

Với nhiều phần mềm thiết kế không tốt, phần xử lý các công việc không liên quan đến vấn đề nghiệp vụ kinh doanh như truy cập tập tin, hạ tầng mạng, cơ sở dữ liệu, \dots được lập trình trong đối tượng nghiệp vụ kinh doanh. Cách này giúp tốc độ hoàn thiện nhanh. Tuy nhiên, cách này làm dự án bị mất đi tính hướng đối tượng khó thay thay đổi, mở rộng hệ thống,\dots Thiết kế hướng miền cung cấp một cách để tổ chức mã nguồn và dễ dàng thích ứng với các yêu cầu thay đổi.

Trong kiến trúc vi dịch vụ, thiết kế hướng miền đảm bảo mỗi dịch vụ được thiết kế phản ánh một phần cụ thể của lĩnh vực kinh doanh. Mỗi dịch vụ được quản lí bởi một nhóm phát triển được hỗ trợ bởi các chuyên gia ngành. Trong thiết kế hướng miền, \emph{chuyên gia ngành (Domain Expert)} là người có kiến thức và hiểu biết sâu sắc về vấn đề đang được hệ thống phần mềm giải quyết. Chuyên gia ngành thể hiện chính xác vấn đề kinh doanh, đóng vai trò là nguồn thông tin cho nhóm phát triển.

\section{Cốt lõi của thiết kế hướng miền}

% Giới thiệu về các mẫu chiến lược và các mẫu kỹ thuật

Thiết kế hướng miền cung cấp 2 loại mẫu:

\begin{itemize}

\item \emph{Các mẫu chiến lược (Strategic Patterns):} Phân chia một miền lớn và phức tạp thành các phần nhỏ hơn với ranh giới được xác định rõ ràng. Giúp phân chia một miền lớn hợp lý.

\item \emph{Các mẫu kỹ thuật (Tactical Patterns):} Hiện thực hóa các khái niệm và qui trình trong thành phần thành các thiết kế hệ thống phần mềm. Giúp hệ thống phù hợp với kinh doanh.

\end{itemize}

\begin{figure}[H]

\centering

\includegraphics[scale = 0.5]{pictures/cac_mau_chien_luoc_va_cac_mau_ky_thuat/main.drawio.png}

\caption{Tổng quan về Strategic Patterns và Tactical Patterns}

\end{figure}

Thiết kế hướng miền cung cấp 2 loại mẫu:

\begin{itemize}

\item \emph{Các mẫu chiến lược (Strategic Patterns):} Phân chia một miền lớn và phức tạp thành các phần nhỏ hơn với ranh giới được xác định rõ ràng. Giúp phân chia một miền lớn hợp lý.

\item \emph{Các mẫu kỹ thuật (Tactical Patterns):} Hiện thực hóa các khái niệm và qui trình trong thành phần thành các thiết kế hệ thống phần mềm. Giúp hệ thống phù hợp với kinh doanh.

\end{itemize}

\begin{figure}[H]

\centering

\includegraphics[scale = 0.5]{pictures/cac_mau_chien_luoc_va_cac_mau_ky_thuat/main.drawio.png}

\caption{Tổng quan về Strategic Patterns và Tactical Patterns}

\end{figure}

% \subsection{Chi tiết về các mẫu chiến lược và các mẫu kỹ thuật}

\section{Các mẫu chiến lược của thiết kế hướng miền}

% \chapter{Các mẫu chiến lược}

Các mẫu chiến lược phân tích nghiệp vụ kinh doanh sau đó đưa ra việc phân chia các thành phần và hiểu mối quan hệ của các thành phần đó. Từ đó, các mẫu chiến lược giúp xác định các thành phần quan trọng của hệ thống, đảm bảo kiến trúc phần mềm phản ánh đúng các yêu cầu kinh doanh. Từ việc phân chia hệ thống thành các thành phần nhỏ, chúng ta có thể tạo ra hệ thống mở rộng dễ dàng, phát triển linh hoạt theo nhu cầu kinh doanh.

Các mẫu chiến lược bao gồm:

\begin{itemize}

% các mục nhỏ ben dưới

% các mục nhỏ ben dưới

% các mục nhỏ ben dưới

% các mục nhỏ ben dưới

% các mục nhỏ ben dưới

% các mục nhỏ ben dưới

\item Muc1

\item Muc2

\item Muc1

\item Muc2

\item Muc1

\item Muc2

\item Muc1

\item Muc2

\end{itemize}

% nội dung trang lớn lên để hết giấy

% nội dung trang lớn lên để hết giấy

% nội dung trang lớn lên để hết giấy

% nội dung trang lớn lên để hết giấy

% nội dung trang lớn lên để hết giấy

\begin{figure}[H]

\centering

\includegraphics[scale = 0.9]{pictures/cac_mau_chien_luoc/temp.png}

\caption{Sơ đồ về các thành phần trong mô hình chiến lược}

\end{figure}

%!<! - - $ Vẽ lại sau: - - >

%!<! - - $ Vẽ lại sau: - - >

%!<! - - $ Vẽ lại sau: - - >

%!<! - - $ Vẽ lại sau: - - >

%!<! - - $ Vẽ lại sau: - - >

%!<! - - $ Vẽ lại sau: - - >

%!<! - - $ Vẽ lại sau: - - >

%!<! - - $ Vẽ lại sau: - - >

%!<! - - $ Vẽ lại sau: - - >

%!<! - - $ Vẽ lại sau: - - >

% \section{Miền phụ (Sub - Domain)}

Một miền lớn được tạo thành từ nhiều \emph{miền phụ (Sub - Domain)}. Trong thực tế, một miền kinh doanh phức tạp không thể có một chuyên gia ngành có kiến thức về tất cả các miền phụ.

\begin{example} Trong miền thương mại điện tử lớn có thể có một số miền phụ sau:

\begin{itemize}

\item \textbf{Miền phụ quản lý hàng tồn kho:} liên quan đến việc quản lý sản phẩm trong kho hàng.

\item \textbf{Miền phụ quản lý khách hàng:} liên quan đến việc quản lý tài khoản khách hàng.

\item \textbf{Miền phụ vận chuyển:} liên quan đến việc quản lý việc vận chuyển giao hàng.

\end{itemize}

\end{example}

% \subsection{Phân loại các miền phụ}

Trong thiết kế hướng miền, có ba loại miền phụ là:

\begin{itemize}

\item Miền phụ chung (Generic Subdomain)

\item Miền phụ cốt lõi (Core Subdomain)

\item Miền phụ hỗ trợ (Supporting Subdomain)

\end{itemize}

% \subsubsection{Miền phụ chung (Generic Subdomain)}

% \input{contents/mien_phu_chung_generic_subdomain}

% \subsubsection{Miền phụ cốt lõi (Core Subdomain)}

% \input{contents/mien_phu_cot_loi_core_subdomain}

% \subsubsection{Miền phụ hỗ trợ (Supporting Subdomain)}

% \input{contents/mien_phu_ho_tro_supporting_subdomain}

% \subsection{Cách xác định các miền phụ}

% \input{contents/cach_xac_dinh_cac_mien_phu}

% \subsection{Áp dụng phân loại miền phụ trong đồ án này}

% \input{contents/ap_dung_phan_loai_mien_phu_trong_do_an_nay}

\section{Các mẫu kỹ thuật của thiết kế hướng miền}

\section{DDD}

\section{DDD}

\section{DDD}

\section{DDD}

\section{DDD}

% \section{Mô hình miền (Domain Models)}

% \input{contents/mo_hinh_mien_domain_models}

% \section{Bối cảnh bị giới hạn (Bounded Context)}

% \input{contents/boi_canh_gioi_han}

% \section{Ngôn ngữ chung (Ubiquitous Language)}

% \input{contents/ngon_ngu_chung}

%@ Tích hợp liên tục

%@ Tích hợp liên tục

%@ Tích hợp liên tục

%@ Tích hợp liên tục

%@ Tích hợp liên tục

% \section{Bản đồ bối cảnh (Context Maps)}

% \input{contents/ban_do_boi_canh}

% \section{Các mối quan hệ bối cảnh bị giới hạn}

% \input{contents/cac_moi_quan_he_boi_canh_gioi_han}

%%%%%%%%%%%%%%%%%%%%%%%%%%%%%%%%%%

%%%%%%%%%%%%%%%%%%%%%%%%%%%%%%%%%%

%%%%%%%%%%%%%%%%%%%%%%%%%%%%%%%%%%

%%%%%%%%%%%%%%%%%%%%%%%%%%%%%%%%%%

%%%%%%%%%%%%%%%%%%%%%%%%%%%%%%%%%%

%%%%%%%%%%%%%%%%%%%%%%%%%%%%%%%%%%

%%%%%%%%%%%%%%%%%%%%%%%%%%%%%%%%%%

% \subsection{Mối quan hệ đối xứng (Symmetric Relationship)}

% \subsubsection{Mô hình riêng biệt (Separate Ways)}

% \input{contents/mo_hinh_rieng_biet_separate_ways}

% \subsubsection{Mô hình hạt nhân chung (Shared Kernel)}

% \input{contents/mo_hinh_hat_nhan_chung_shared_kernel}

% \subsection{Mối quan hệ bất đối xứng (Asymmetric Relationship)}

% \input{contents/moi_quan_he_bat_doi_xung_asymmetric_relationship}

% \subsubsection{Mô hình khách hàng - nhà cung cấp (Customer - Supplier)}

% \input{contents/mo_hinh_khach_hang_nha_cung_cap_customer_supplier}

% \subsubsection{Mô hình tuân thủ (Conformist)}

% \input{contents/mo_hinh_tuan_thu_conformist}

% \subsubsection{Mô hình chống đổ vỡ (Anti Corruption Layer)}

% \input{contents/mo_hinh_chong_do_vo_anti_corruption_layer}

% \subsection{Mối quan hệ 1 - nhiều (One to Many Relationship)}

% \subsubsection{Dịch vụ máy chủ mở (Open Host Service)}

% \input{contents/dich_vu_may_chu_mo_open_host_srv}

% \subsubsection{Ngôn ngữ được xuất bản (Published Language)}

% \input{contents/ngon_ngu_duoc_xuat_ban_published_language}

%%%%%%%%%%%%%%%%%%%%%%%%%%%%%%%%%%

%%%%%%%%%%%%%%%%%%%%%%%%%%%%%%%%%%

%%%%%%%%%%%%%%%%%%%%%%%%%%%%%%%%%%

%%%%%%%%%%%%%%%%%%%%%%%%%%%%%%%%%%

%%%%%%%%%%%%%%%%%%%%%%%%%%%%%%%%%%

%%%%%%%%%%%%%%%%%%%%%%%%%%%%%%%%%%

%%%%%%%%%%%%%%%%%%%%%%%%%%%%%%%%%%

%%%%%%%%%%%%%%%%%%%%%%%%%%%%%%%%%%

% \section{Áp dụng về các mối quan hệ bối cảnh bị giới hạn}

%! Hướng dẫn 6/6

%%%%%%%%%%%%%%%%%%%%%%%%%%%%%%%%%%

%%%%%%%%%%%%%%%%%%%%%%%%%%%%%%%%%%

%%%%%%%%%%%%%%%%%%%%%%%%%%%%%%%%%%

%%%%%%%%%%%%%%%%%%%%%%%%%%%%%%%%%%

%%%%%%%%%%%%%%%%%%%%%%%%%%%%%%%%%%

%%%%%%%%%%%%%%%%%%%%%%%%%%%%%%%%%%

%%%%%%%%%%%%%%%%%%%%%%%%%%%%%%%%%%

%%%%%%%%%%%%%%%%%%%%%%%%%%%%%%%%%%

\chapter{Các mẫu kỹ thuật}

% \input{contents/cac_mau_ky_thuat}

\section{Đối tượng miền (Domain Object)}

% \input{contents/cac_doi_tuong_mien_domain_object}

\subsection{Đối tượng thực thể (Entities Objects)}

\input{contents/doi_tuong_thuc_the_entities_objects}

\subsection{Đối tượng giá trị (Value Objects)}

% Đối tượng miền hoặc đối tượng giá trị, là đối tượng đại diện cho một đặc điểm của miền mà không có nhận dạng riêng.

% \input{contents/doi_tuong_gia_tri_value_objects}

\subsection{Miền dịch vụ (Service Domain)}

% \input{contents/mien_dich_vu_srv}

%! Hướng dẫn 7/4

%! Hướng dẫn 7/5

% \subsubsection{xxxxxxx}

% \chapter*{\centering MỞ ĐẦU}
\addcontentsline{toc}{chapter}{MỞ ĐẦU}
\section*{Lý do chọn đề tài}
% \input{contents/ly_do_chon_de_tai}
\section*{Đối tượng và phạm vi nghiên cứu}
% \input{contents/doi_tuong_va_pham_vi_nghien_cuu}
\section*{Tóm tắt nội dung đồ án}
% \input{contents/tom_tat_noi_dung_do_an}

%%%%%%%%%%%%%%%%%%%%%%%%%%%%%%%%%%

\end{document} % Kết thúc

% kết luận, tài liệu tham khảo

% \bibliographystyle{plain}

% \bibliography{references}

%%%%%%%%%%%%%%%%%%%%%%%%%%%%%%%%%%

% % %! Aggregates/ /

% % Tổng hợp là đối tượng kinh doanh trung tâm trong Bối cảnh bị giới hạn của chúng ta và xác định phạm vi nhất quán trong bối cảnh bị giới hạn đó.

% % Tổng hợp = Mã định danh chính của Bối cảnh bị giới hạn của chúng ta

% \subsubsection{xxxxxxx}

% % \chapter*{\centering MỞ ĐẦU}
\addcontentsline{toc}{chapter}{MỞ ĐẦU}
\section*{Lý do chọn đề tài}
% \input{contents/ly_do_chon_de_tai}
\section*{Đối tượng và phạm vi nghiên cứu}
% \input{contents/doi_tuong_va_pham_vi_nghien_cuu}
\section*{Tóm tắt nội dung đồ án}
% \input{contents/tom_tat_noi_dung_do_an}

% \end{document} % kết thúc

% Yêu cầu nghiệp vụ của từng sub

% %

% Sơ đồ if else Đ S

% %

% sub trước model

% %

%%%%%%%%%%%%%%%%%%%%%%%%%%%%%%%%%%%%%

\end{document}

\section{xxxxxxx}

\subsection{xxxxxxx}

\subsubsection{xxxxxxx}

\chapter*{\centering MỞ ĐẦU}
\addcontentsline{toc}{chapter}{MỞ ĐẦU}
\section*{Lý do chọn đề tài}
% \input{contents/ly_do_chon_de_tai}
\section*{Đối tượng và phạm vi nghiên cứu}
% \input{contents/doi_tuong_va_pham_vi_nghien_cuu}
\section*{Tóm tắt nội dung đồ án}
% \input{contents/tom_tat_noi_dung_do_an}

