% \documentclass{article}
% \documentclass{book}
\documentclass{report} % Chọn cỡ chữ
\usepackage{init}
\usepackage{vvn}
\begin{document} % Bắt đầu
%%%%%%%%%%%%%%%%%%%%%%%%%%%%%%
% \input{contents/trang_bia}

\input{contents/trang_bia}
%%%%%%%%%%%%%%%%%%%%%%%%%%%%%%

\begin{center}

    {\bfseries NHẬN XÉT CỦA GIẢNG VIÊN HƯỚNG DẪN}
    
    \end{center}
    
    \begin{enumerate}
    
    \item Mục đích và nội dung của đồ án:
    
    \vspace{20ex}
    
    \item Kết quả đạt được:
    
    \vspace{20ex}
    
    \item Ý thức làm việc của sinh viên:
    
    \vspace{20ex}
    
    \end{enumerate}
    
    \hspace{0.4\textwidth}\begin{minipage}{0.5\textwidth}
    
    \noindent\begin{center}
    
    \textit{Hà Nội, \today} \\
    
    \textbf{Giảng viên hướng dẫn} \\
    
    \textit{(Ký và ghi rõ họ tên)}
    
    \vspace{2cm}
    
    \textbf{TS. Vũ Thành Nam}
    
    \end{center}
    
    \end{minipage}
    
    \pagestyle{empty}
    
%%%%%%%%%%%%%%%%%%%%%%%%%%%%%%
    

\includepdf[pages = -]{contents/bao_cao_tien_do_1.pdf}

\includepdf[pages = -]{contents/bao_cao_tien_do_2.pdf}
%%%%%%%%%%%%%%%%%%%%%%%%%%%%%%

\renewcommand*\contentsname{\centering MỤC LỤC}

\tableofcontents

\setcounter{page}{0}

%%%%%%%%%%%%%%%%%%%%%%%%%%%%%%


\input{contents/loi_cam_on}

%%%%%%%%%%%%%%%%%%%%%%%%%%%%%%
\chapter*{\centering DANH SÁCH BẢNG}

\addcontentsline{toc}{chapter}{DANH SÁCH BẢNG}

\makeatletter

\renewcommand\listoftables{

\@starttoc{lot}

}

\makeatother

\listoftables



%%%%%%%%%%%%%%%%%%%%%%%%%%%%%%
\chapter*{\centering DANH SÁCH HÌNH ẢNH}

\addcontentsline{toc}{chapter}{DANH SÁCH HÌNH ẢNH}

\makeatletter

\renewcommand\listoffigures{

\@starttoc{lof}

}

\makeatother

\listoffigures



%%%%%%%%%%%%%%%%%%%%%%%%%%%%%%
\chapter*{\centering DANH SÁCH CÁC CỤM TỪ VIẾT TẮT}

\addcontentsline{toc}{chapter}{DANH SÁCH CÁC CỤM TỪ VIẾT TẮT}

% @sau

% @sau

% @sau

% @sau

% @sau

% @sau

% @sau

% @sau

% @sau

% @sau

% @sau

% @sau

% @sau

% @sau

% @sau

% @sau

% @sau

% @sau

% @sau

% @sau

% @sau

% @sau

% @sau

% @sau

% @sau

% @sau

% @sau

% @sau

% @sau

% @sau

\begin{table}[h]

\centering

\begin{tabular}{|c|c|c|c|}

\hline

STT & Từ viết tắt & Từ viết đầy đủ & Mô tả \\

\hline

Dong1 & Dong1 & Cot1 & Cot2 \\

\hline

Dong2 & Dong2 & Cot1 & Cot2 \\

\hline

\end{tabular}

\end{table}

% API; Application Programming Interface; Giao diện lập trình ứng dụng

% CI/CD; Continuous Integration (CI) and Continuous Delivery (CD) ; Quá trình tích hợp và chuyển giao liên tục

% thiết kế hướng miền ; thiết kế hướng miền; Kỹ thuật thiết kế theo hướng miền

% DI; Dependency Injection; Cơ chế tiêm sự phụ thuộc giữa các đối tượng

% HTTP; Hypertext Transfer Protocol; Giao thức truyền tải siêu văn bản

% JSON; JavaScript Object Notation; Một kiểu dữ liệu mở rộng của JavaScript

% ORM; Object Relational Mapping; Một kỹ thuật ánh xạ các đối tượng lập trình với từng bảng trong CSDL quan hệ

% Cơ sở dữ liệu ; CSDL ;

% Tạo (Create), Đọc (Read), Sửa (Update), Xóa (Delete) ; CRUD ;

% Kubernetes ; K8s ; kubernetes

% Số điện thoại ; SĐT ;

% UML

% MVC; Model View Controller; Một mẫu thiết kế ứng dụng

% SQL

SOA; Service Oriented Architecture; Kiến trúc hướng dịch vụ

SOAP; Simple Object Access Protocol; Một giao thức để truy cập dịch vụ web

SPA; Single Page Application; Kiểu ứng dụng một trang

REST; Representational State Transfer; Một tiêu chuẩn thiết kế các API sử dụng cho các dịch vụ web

URL; Uniform Resource Locator ; Địa chỉ định vị tài nguyên trên Internet

XML; Extensible Markup Language; Ngôn ngữ đánh dấu mở rộng

% TCT ; TCT ;

Người nộp thuế ; NNT ;

Mã số thuế ; MST ;

Hóa đơn điện tử ; HĐĐT ;

Cơ quan thuế ; CQT ;

Công nghệ thông tin ; CNTT ;



%%%%%%%%%%%%%%%%%%%%%%%%%%%%%%
\chapter*{\centering DANH SÁCH CÁC THUẬT NGỮ}

\addcontentsline{toc}{chapter}{DANH SÁCH CÁC THUẬT NGỮ}

% @sau

% @sau

% @sau

% @sau

% @sau

% @sau

% @sau

% @sau

% @sau

% @sau

% @sau

% @sau

% @sau

% @sau

% @sau

% @sau

% @sau

% @sau

% @sau

% @sau

% @sau

% @sau

% @sau

% @sau

% @sau

% @sau

% @sau

% @sau

\begin{table}[h]

\centering

\begin{tabular}{|c|c|c|}

\hline

STT & Tiếng Anh & Tiếng Việt \\

\hline

Dong1 & Dong1 & Cot2 \\

\hline

Dong2 & Dong2 & Cot2 \\

\hline

\end{tabular}

\end{table}

% kiến trúc nguyên khối, kiến trúc nguyên khối

% kiến trúc nguyên khối, kiến trúc nguyên khối

% kiến trúc vi dịch, kiến trúc vi dịch

% kiến trúc vi dịch, kiến trúc vi dịch

% kiến trúc vi dịch, kiến trúc vi dịch

% kiến trúc vi dịch, kiến trúc vi dịch

% thiết kế hướng miền, thiết kế hướng miền

% thiết kế hướng miền, thiết kế hướng miền

1 thiết kế hướng miền

Thiết kế hướng lĩnh vực

2 Domain (không dịch)

3 Abstraction Trừu tượng

4 chuyên gia ngành

%%%%%%%%%%%%%%%%%%%%%%%%%%%%%%
\chapter*{\centering MỞ ĐẦU}

\addcontentsline{toc}{chapter}{MỞ ĐẦU}

\section*{Lý do chọn đề tài}

\input{contents/ly_do_chon_de_tai}

\section*{Đối tượng và phạm vi nghiên cứu}

\input{contents/doi_tuong_va_pham_vi_nghien_cuu}

\section*{Tóm tắt nội dung đồ án}

\input{contents/tom_tat_noi_dung_do_an}

%%%%%%%%%%%%%%%%%%%%%%%%%%%%%%
%%%%%%%%%%%%%%%%%%%%%%%%%%%%%%
\chapter*{\centering MỞ ĐẦU}
\addcontentsline{toc}{chapter}{MỞ ĐẦU}
\section*{Lý do chọn đề tài}
% \input{contents/ly_do_chon_de_tai}
\section*{Đối tượng và phạm vi nghiên cứu}
% \input{contents/doi_tuong_va_pham_vi_nghien_cuu}
\section*{Tóm tắt nội dung đồ án}
% \input{contents/tom_tat_noi_dung_do_an}
https://authorization-service-api.web.cern.ch/swagger/index.html
\section{Giới thiệu về thiết kế hướng miền (Domain Driven Design)}
Thiết kế hướng miền được Eric Evans giới thiệu trong cuốn sách \emph{"Domain Driven Design: Tackling Complexity in the Heart of Software"}. \emph{Thiết kế hướng miền (Domain Driven Design)} là một hướng tiếp cận thiết kế phần mềm tập trung vào việc hiểu rõ và mô hình hóa lĩnh vực kinh doanh của một tổ chức. Thiết kế hướng miền nhấn mạnh việc sử dụng lĩnh vực nghiệp vụ kinh doanh để thảo luận và đề xuất giải pháp đáp ứng nhu cầu.

Hệ thống phần mềm được tạo ra để xử lý công việc trong cuộc sống hiện đại. Việc phát triển hệ thống liên kết chặt chẽ với một số khía cạnh cụ thể trong cuộc sống của chúng ta. Trong thiết kế hướng miền, \emph{miền (Domain)} đề cập đến phạm vi kiến thức và vấn đề cụ thể mà hệ thống xử lý.

\begin{itemize}

\item Về góc độ kinh doanh: Miền đại diện cho một lĩnh vực hoặc ngành mà doanh nghiệp hoạt động.

\item Về góc độ hệ thống: Miền có thể coi là đại diện cho không gian vấn đề của hệ thống.

\end{itemize}

\begin{example} \emph{Trong đồ án này, miền được xác định là bài toán giải pháp hóa đơn điện tử.}

\end{example}

Với nhiều phần mềm thiết kế không tốt, phần xử lý các công việc không liên quan đến vấn đề nghiệp vụ kinh doanh như truy cập tập tin, hạ tầng mạng, cơ sở dữ liệu, \dots được lập trình trong đối tượng nghiệp vụ kinh doanh. Cách này giúp tốc độ hoàn thiện nhanh. Tuy nhiên, cách này làm dự án bị mất đi tính hướng đối tượng khó thay thay đổi, mở rộng hệ thống,\dots Thiết kế hướng miền cung cấp một cách để tổ chức mã nguồn và dễ dàng thích ứng với các yêu cầu thay đổi.

Trong kiến trúc vi dịch vụ, thiết kế hướng miền đảm bảo mỗi dịch vụ được thiết kế phản ánh một phần cụ thể của lĩnh vực kinh doanh. Mỗi dịch vụ được quản lí bởi một nhóm phát triển được hỗ trợ bởi các chuyên gia ngành. Trong thiết kế hướng miền, \emph{chuyên gia ngành (Domain Expert)} là người có kiến thức và hiểu biết sâu sắc về vấn đề đang được hệ thống phần mềm giải quyết. Chuyên gia ngành thể hiện chính xác vấn đề kinh doanh, đóng vai trò là nguồn thông tin cho nhóm phát triển.
\subsection{Giới thiệu về các mẫu chiến lược và các mẫu kỹ thuật}
\input{contents/gioi_thieu_ve_cac_mau_chien_luoc_va_cac_mau_ky_thuat}

% @ Thêm mục , chia

%%%%%%%%%%%%%%%%%%%%%%%%%%%%%%
vvn20206205
\end{document}

% \input{contents/chi_tiet_va_ap_dung_thiet_ke_huong_mien}
% \subsection{Chi tiết về các mẫu chiến lược và các mẫu kỹ thuật}

% \chapter{Các mẫu chiến lược}

% \input{contents/cac_mau_chien_luoc}

% \newpage

% \newpage

% \newpage

% \newpage

% \newpage

% \section{Miền phụ (Sub - Domain)}

% \input{contents/mien_phu_sub_domain}

% \subsection{Phân loại các miền phụ}

% \input{contents/phan_loai_cac_mien_phu}

% \subsubsection{Miền phụ chung (Generic Subdomain)}

% \input{contents/mien_phu_chung_generic_subdomain}

% \subsubsection{Miền phụ cốt lõi (Core Subdomain)}

% \input{contents/mien_phu_cot_loi_core_subdomain}

% \subsubsection{Miền phụ hỗ trợ (Supporting Subdomain)}

% \input{contents/mien_phu_ho_tro_supporting_subdomain}

% \newpage

% \newpage

% \newpage

% \newpage

% \newpage

% \newpage

% \newpage

% \newpage

% \newpage

% \newpage

% \newpage

% \newpage

% \newpage

% \newpage

% \newpage

% \newpage

% \newpage

% \newpage

% \subsection{Cách xác định các miền phụ}

% \input{contents/cach_xac_dinh_cac_mien_phu}

% \newpage

% \newpage

% \newpage

% \newpage

% \newpage

% \newpage

% \newpage

% \newpage

% \newpage

% \newpage

% \newpage

% \newpage

% \newpage

% \newpage

% \subsection{Áp dụng phân loại miền phụ trong đồ án này}

% \input{contents/ap_dung_phan_loai_mien_phu_trong_do_an_nay}

% \section{Mô hình miền (Domain Models)}

% \input{contents/mo_hinh_mien_domain_models}

% \section{Bối cảnh bị giới hạn (Bounded Context)}

% \input{contents/boi_canh_gioi_han}

% \section{Ngôn ngữ chung (Ubiquitous Language)}

% \input{contents/ngon_ngu_chung}

%@ Tích hợp liên tục

%@ Tích hợp liên tục

%@ Tích hợp liên tục

%@ Tích hợp liên tục

%@ Tích hợp liên tục

% \section{Bản đồ bối cảnh (Context Maps)}

% \input{contents/ban_do_boi_canh}

% \section{Các mối quan hệ bối cảnh bị giới hạn}

% \input{contents/cac_moi_quan_he_boi_canh_gioi_han}

%%%%%%%%%%%%%%%%%%%%%%%%%%%%%%%%%%

%%%%%%%%%%%%%%%%%%%%%%%%%%%%%%%%%%

%%%%%%%%%%%%%%%%%%%%%%%%%%%%%%%%%%

%%%%%%%%%%%%%%%%%%%%%%%%%%%%%%%%%%

%%%%%%%%%%%%%%%%%%%%%%%%%%%%%%%%%%

%%%%%%%%%%%%%%%%%%%%%%%%%%%%%%%%%%

%%%%%%%%%%%%%%%%%%%%%%%%%%%%%%%%%%

% \subsection{Mối quan hệ đối xứng (Symmetric Relationship)}

% \subsubsection{Mô hình riêng biệt (Separate Ways)}

% \input{contents/mo_hinh_rieng_biet_separate_ways}

% \subsubsection{Mô hình hạt nhân chung (Shared Kernel)}

% \input{contents/mo_hinh_hat_nhan_chung_shared_kernel}

% \subsection{Mối quan hệ bất đối xứng (Asymmetric Relationship)}

% \input{contents/moi_quan_he_bat_doi_xung_asymmetric_relationship}

% \subsubsection{Mô hình khách hàng - nhà cung cấp (Customer - Supplier)}

% \input{contents/mo_hinh_khach_hang_nha_cung_cap_customer_supplier}

% \subsubsection{Mô hình tuân thủ (Conformist)}

% \input{contents/mo_hinh_tuan_thu_conformist}

% \subsubsection{Mô hình chống đổ vỡ (Anti Corruption Layer)}

% \input{contents/mo_hinh_chong_do_vo_anti_corruption_layer}

% \subsection{Mối quan hệ 1 - nhiều (One to Many Relationship)}

% \subsubsection{Dịch vụ máy chủ mở (Open Host Service)}

% \input{contents/dich_vu_may_chu_mo_open_host_srv}

% \subsubsection{Ngôn ngữ được xuất bản (Published Language)}

% \input{contents/ngon_ngu_duoc_xuat_ban_published_language}

%%%%%%%%%%%%%%%%%%%%%%%%%%%%%%%%%%

%%%%%%%%%%%%%%%%%%%%%%%%%%%%%%%%%%

%%%%%%%%%%%%%%%%%%%%%%%%%%%%%%%%%%

%%%%%%%%%%%%%%%%%%%%%%%%%%%%%%%%%%

%%%%%%%%%%%%%%%%%%%%%%%%%%%%%%%%%%

%%%%%%%%%%%%%%%%%%%%%%%%%%%%%%%%%%

%%%%%%%%%%%%%%%%%%%%%%%%%%%%%%%%%%

%%%%%%%%%%%%%%%%%%%%%%%%%%%%%%%%%%

% \section{Áp dụng về các mối quan hệ bối cảnh bị giới hạn}

%! Hướng dẫn 6/6

%%%%%%%%%%%%%%%%%%%%%%%%%%%%%%%%%%

%%%%%%%%%%%%%%%%%%%%%%%%%%%%%%%%%%

%%%%%%%%%%%%%%%%%%%%%%%%%%%%%%%%%%

%%%%%%%%%%%%%%%%%%%%%%%%%%%%%%%%%%

%%%%%%%%%%%%%%%%%%%%%%%%%%%%%%%%%%

%%%%%%%%%%%%%%%%%%%%%%%%%%%%%%%%%%

%%%%%%%%%%%%%%%%%%%%%%%%%%%%%%%%%%

%%%%%%%%%%%%%%%%%%%%%%%%%%%%%%%%%%

\chapter{Các mẫu kỹ thuật}
% \input{contents/cac_mau_ky_thuat}

\section{Đối tượng miền (Domain Object)}
% \input{contents/cac_doi_tuong_mien_domain_object}
\subsection{Đối tượng thực thể (Entities Objects)}
\input{contents/doi_tuong_thuc_the_entities_objects}
\subsection{Đối tượng giá trị (Value Objects)}
% Đối tượng miền hoặc đối tượng giá trị, là đối tượng đại diện cho một đặc điểm của miền mà không có nhận dạng riêng.
% \input{contents/doi_tuong_gia_tri_value_objects}
\subsection{Miền dịch vụ (Service Domain)}
% \input{contents/mien_dich_vu_srv}

%! Hướng dẫn 7/4

%! Hướng dẫn 7/5

% \subsubsection{xxxxxxx}

% \chapter*{\centering MỞ ĐẦU}
\addcontentsline{toc}{chapter}{MỞ ĐẦU}
\section*{Lý do chọn đề tài}
% \input{contents/ly_do_chon_de_tai}
\section*{Đối tượng và phạm vi nghiên cứu}
% \input{contents/doi_tuong_va_pham_vi_nghien_cuu}
\section*{Tóm tắt nội dung đồ án}
% \input{contents/tom_tat_noi_dung_do_an}

%%%%%%%%%%%%%%%%%%%%%%%%%%%%%%%%%%

\end{document} % Kết thúc

% kết luận, tài liệu tham khảo

% \bibliographystyle{plain}
% \bibliography{references}
%%%%%%%%%%%%%%%%%%%%%%%%%%%%%%%%%%

% % %! Aggregates/ /

% % Tổng hợp là đối tượng kinh doanh trung tâm trong Bối cảnh bị giới hạn của chúng ta và xác định phạm vi nhất quán trong bối cảnh bị giới hạn đó.

% % Tổng hợp = Mã định danh chính của Bối cảnh bị giới hạn của chúng ta

% \subsubsection{xxxxxxx}

% % \chapter*{\centering MỞ ĐẦU}
\addcontentsline{toc}{chapter}{MỞ ĐẦU}
\section*{Lý do chọn đề tài}
% \input{contents/ly_do_chon_de_tai}
\section*{Đối tượng và phạm vi nghiên cứu}
% \input{contents/doi_tuong_va_pham_vi_nghien_cuu}
\section*{Tóm tắt nội dung đồ án}
% \input{contents/tom_tat_noi_dung_do_an}

% \end{document} % kết thúc

% Yêu cầu nghiệp vụ của từng sub

% %

% Sơ đồ if else Đ S

% %

% sub trước model

% %

%%%%%%%%%%%%%%%%%%%%%%%%%%%%%%%%%%%%%

\end{document}

\section{xxxxxxx}

\subsection{xxxxxxx}

\subsubsection{xxxxxxx}

\chapter*{\centering MỞ ĐẦU}
\addcontentsline{toc}{chapter}{MỞ ĐẦU}
\section*{Lý do chọn đề tài}
% \input{contents/ly_do_chon_de_tai}
\section*{Đối tượng và phạm vi nghiên cứu}
% \input{contents/doi_tuong_va_pham_vi_nghien_cuu}
\section*{Tóm tắt nội dung đồ án}
% \input{contents/tom_tat_noi_dung_do_an}

% phải có CQRS (Phân chia trách nhiệm truy vấn lệnh)

CQRS là một mẫu kiến trúc riêng biệt có thể được sử dụng kết hợp với thiết kế hướng miền để đạt được những lợi ích nhất định, chẳng hạn như cải thiện hiệu suất và khả năng mở rộng. Tuy nhiên, nó không phải là một yêu cầu để triển khai thiết kế hướng miền.

% phải có event

Ngôn ngữ chung (Ubiquitous Language)

%%%%%%%%%%%%%%%%%%%%%%%%%%%%%%%%%%%%%

\end{document} % kết thúc

Cách tiếp cận này nhấn mạnh tính mô - đun, tính linh hoạt và khả năng phục hồi, cho phép các nhóm làm việc đồng thời trên các phần khác nhau của hệ thống và cho phép phát hành nhanh hơn và thường xuyên hơn. Các vi dịch vụ thường dựa vào các giao thức truyền thông nhẹ, chẳng hạn như REST và thường được triển khai bằng các công nghệ chứa trong bộ chứa như Docker và Kubernetes.

\subsubsection{DevOps Ứng dụng, áp dụng, liên quan,....}

\subsubsection{CI/CD}

\subsubsection{Docker}

\subsubsection{Kubernetes}

dícovery

api gateway
